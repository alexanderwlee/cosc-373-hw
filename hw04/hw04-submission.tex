\documentclass{amsart}

\usepackage[vlined,linesnumbered]{algorithm2e}
\usepackage{enumitem}

\title{COSC-373: Homework 4}
\author{Lee Jiaen}
\author{Hyery Yoo}
\author{Alexander Lee}

\theoremstyle{definition}
\newtheorem{question}{Question}

\begin{document}

\maketitle

\begin{question}
  Each node $v$ executes as follows.

  Each round:
  \begin{enumerate}
    \item Pick a uniformly random color $c(v)$ from the range $1, 2, \ldots, 2
      \deg(v)$
    \item Send message $c(v)$ to all neighbors
    \item If $c(v) \neq c(u)$ for all neighbors $u$, output $c(v)$ and halt
  \end{enumerate}
\end{question}

(Runtime) To show that the procedure produces a proper $2 \Delta$ coloring of
$G$ in $O(\log n)$, first define the random variable $X_v$ for whether a given
node $v$ halts in a given round. Specifically, $X_v$ is 1 if node $v$ halts in
the given round, and 0 otherwise. The probability $\mathbb{P}(X_v)$ that node
$v$ halts in a given round, is the probability $\frac{1}{2 \deg(v)}$ that node $v$
chooses color $c(v)$ from the $2 \deg(v)$ colors times the probability
$\frac{\deg(v)}{2 \deg(v) - 1}$ that each of node $v$'s $\deg(v)$ neighbors do
not choose the color $c(v)$ from the $2 \deg(v)$ colors. That is,
\begin{align*}
  \mathbb{P}(X_v) &= \frac{1}{2 \deg(v)} \cdot \frac{\deg(v)}{2 \deg(v) - 1} \\
  &= \frac{1}{4 \deg(v) - 2} \enspace.
\end{align*}
It follows that the expected value $\mathbb{E}(X_v)$ of node $v$ halting in a
given round is
\begin{align*}
  \mathbb{E}(X_v) &= 1 \cdot \frac{1}{4 \deg(v) - 2} + 0 \cdot (1 - \frac{1}{4
  \deg(v) - 2}) \\
  &= \frac{1}{4 \deg(v) - 2} \enspace.
\end{align*}
Next, define the random variable $X$ for the number of nodes that halt in a
given round. Given the set of nodes $V$ for graph $G$, we have that
\[
  X = \sum_{v \in V} X_v \enspace.
\]
Taking the expected value on both sides, we get
\begin{align*}
  \mathbb{E}(X) &= \mathbb{E}\left(\sum_{v \in V} X_v\right) \\
  &= \sum_{v \in V} \mathbb{E}(X_v) \\
  &= \sum_{v \in V} \frac{1}{4 \deg(v) - 2} \\
  &\ge \sum_{v \in V} \frac{1}{4 \Delta - 2} \\
  &> \sum_{v \in V} \frac{1}{4 \Delta} \\
  &= \frac{1}{4 \Delta} n \enspace.
\end{align*}
In expectation, $\frac{1}{4 \Delta}$ nodes thus halt in each round, and the
procedure therefore produces a proper $2 \Delta$ coloring of $G$ in $O(\log n)$
rounds with high probability.

\newpage

\begin{question}
  TODO
\end{question}

\newpage

\begin{question}
  TODO
\end{question}

\end{document}
