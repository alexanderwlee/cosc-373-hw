\documentclass{amsart}

\usepackage[vlined,linesnumbered]{algorithm2e}
\usepackage{amssymb}
\usepackage{enumitem}

\title{COSC-373: Homework 4}
\author{Lee Jiaen}
\author{Hyery Yoo}
\author{Alexander Lee}

\theoremstyle{definition}
\newtheorem{question}{Question}

\begin{document}

\maketitle

\begin{question}
  Each node $v$ executes as follows.

  Each round:
  \begin{enumerate}
    \item Pick a uniformly random color $c(v)$ from the range $1, 2, \ldots, 2
      \deg(v)$
    \item Send message $c(v)$ to all neighbors
    \item If $c(v) \neq c(u)$ for all neighbors $u$, output $c(v)$ and halt
  \end{enumerate}

  (Correctness) At each round a node $v$ picks a color $c(v)$ uniformly at
  random from the range $1, 2, \ldots, 2 \deg(v)$. Then, node $v$ sends its
  chosen color $c(v)$ to all its neighbors. Similarly, node $v$ would also
  receive colors from its active neighbors. If one of $v$'s neighbors has
  already halted, $v$ uses the last color it received from the neighbor. For
  each neighbor $u$ of node $v$, if $c(v) \neq c(u)$, the condition that
  neighboring vertices are assigned different colors is satisfied, so node $v$
  outputs $c(v)$ and halts. Therefore, this protocol produces a proper $2
  \Delta$ coloring of $G$.

  (Congestion) Since the protocol only requires nodes to send colors from the
  range $1, 2, \ldots, 2 \Delta$ to neighboring nodes, only messages of size
  $O(1)$ bits are being sent. Thus, the protocol is in the CONGEST model.

  (Runtime) To show that the protocol produces a proper $2 \Delta$ coloring of
  $G$ in $O(\log n)$ rounds, first define the random variable $X_v$ for whether
  a node $v$ halts in a given round. Specifically, $X_v$ is 1 if node $v$ halts
  in the given round, and 0 otherwise. For a given round, since node $v$ has
  $\deg(v)$ neighbors, its neighbors can choose at most $\deg(v)$ distinct
  colors from each other. We also know that node $v$ chooses uniformly at random
  from $2 \deg(v)$ colors. As such, the probability that node $v$ chooses the
  same color as one of its neighbors, is at most $\frac{\deg(v)}{2 \deg(v)} =
  \frac{1}{2}$. It follows that the probability $\mathbb{P}(X_v)$ that node $v$
  halts in a given round (i.e., choosing a different color from its neighbors)
  is
  \begin{align*}
    \mathbb{P}(X_v) &\ge 1 - \frac{1}{2} \\
    &= \frac{1}{2}.
  \end{align*}
  Hence, the expected value $\mathbb{E}(X_v)$ of node $v$ halting in a given
  round is
  \begin{align*}
    \mathbb{E}(X_v) &\ge 1 \cdot \frac{1}{2} + 0 \cdot \frac{1}{2} \\
    &= \frac{1}{2}.
  \end{align*}
  Next, define the random variable $X$ for the number of nodes that halt in a
  given round. Given the set of nodes $V$ for graph $G$, we have that
  \[
    X = \sum_{v \in V} X_v \enspace.
  \]
  Taking the expected value on both sides, we get
  \begin{align*}
    \mathbb{E}(X) &= \mathbb{E}\left(\sum_{v \in V} X_v\right) \\
    &= \sum_{v \in V} \mathbb{E}(X_v) \\
    &\ge \sum_{v \in V} \frac{1}{2} \\
    &= \frac{1}{2} n
  \end{align*}
  Thus, in expectation, at least $\frac{1}{2}$ of all nodes halt in each round.
  If $n = n_0, n_1, \ldots, n_k$ are the number of active nodes after rounds
  $0, 1, \ldots, k$, we have that
  \[
    \mathbb{E}(n_k) \le \frac{1}{2^k} \cdot n.
  \]
  Taking $k = 4 \log n$, we get
  \begin{align*}
    \mathbb{E}(n_k) &\le \frac{1}{2^{4 \log n}} \cdot n \\
    &= \frac{1}{n^4} \cdot n \\
    &= \frac{1}{n^3}.
  \end{align*}
  By Markov's Inequality, it follows that $\mathbb{P}(n_k \ge 1) \le
  \frac{1}{n^3}$. That is, after $4 \log n$ rounds, the probability that there
  are one or more active nodes is at most $\frac{1}{n^3}$. Therefore, the
  protocol produces a proper $2 \Delta$ coloring of $G$ in $O(\log n)$ rounds
  with high probability.
\end{question}

\newpage

\begin{question}
  For $F = \{(A, A^c) \mid A \subseteq \{1, 2, \ldots, N\}\}$ to be a fooling
  set for function DISJ, we must prove the following conditions:
  \begin{enumerate}
    \item For all $i, j$, we have $\text{DISJ}(A_i, A_i^c) = \text{DISJ}(A_j,
      A_j^c)$; and
    \item For all $i \neq j$, we have $\text{DISJ}(A_i, A_i^c) \neq
      \text{DISJ}(A_i, A_j^c)$ or  $\text{DISJ}(A_i, A_i^c) \neq
      \text{DISJ}(A_j, A_i^c)$.
  \end{enumerate}

  To prove condition (1), let $(A_i, A_i^c) \in F$. Since $A_i^c$ is the
  complement of $A_i$, it follows that $A_i$ and $A_i^c$ are disjoint because
  $A_i \cap A_i^c = \emptyset$. Thus, $\text{DISJ}(A_i, A_i^c) = 1$. Notice that
  $(A_i, A_i^c)$ is any arbitrary element in $F$. Therefore, for all $i, j$
  where $(A_i, A_i^c), (A_j, A_j^c) \in F$, we have that $\text{DISJ}(A_i,
  A_i^c) = 1 = \text{DISJ}(A_j, A_j^c)$.

  To prove condition (2), let $(A_i, A_i^c), (A_j, A_j^c) \in F$, where $i \neq
  j$. Three cases are possible.

  Case 1: $A_j \subset A_i$ (and $A_j \not\subset A_i^c$). This implies that
  $A_i \cap A_j^c \neq \emptyset$ and $A_j \cap A_i^c = \emptyset$. Thus, we
  have that $\text{DISJ}(A_i, A_j^c) = 0$ and $\text{DISJ}(A_j, A_i^c) = 1$.
  Since $\text{DISJ}(A_i, A_i^c) = 1 \neq 0 = \text{DISJ}(A_i, A_j^c)$,
  condition (2) is satisfied for this case.

  Case 2: $A_j \subset A_i^c$ (and $A_j \not\subset A_i$). This implies that
  $A_i \cap A_j^c \neq \emptyset$ and $A_j \cap A_i^c \neq \emptyset$. Hence, we
  have that $\text{DISJ}(A_i, A_j^c) = 0$ and $\text{DISJ}(A_j, A_i^c) = 0$.
  Since $\text{DISJ}(A_i, A_i^c) = 1 \neq 0 = \text{DISJ}(A_i, A_j^c)$,
  condition (2) is also satisfied for this case.

  Case 3: $A_j \not\subset A_i$ and  $A_j \not\subset A_i^c$. This implies that
  $A_i \cap A_j^c \neq \emptyset$ and $A_j \cap A_i^c \neq \emptyset$. It
  follows that $\text{DISJ}(A_i, A_j^c) = 0$ and $\text{DISJ}(A_j, A_i^c = 0)$.
  Since $\text{DISJ}(A_i, A_i^c) = 1 \neq 0 = \text{DISJ}(A_i, A_j^c)$,
  condition (2) is satisfied for this case, and the proof is complete.
\end{question}

\newpage

\begin{question}
  The table below shows the result of GT for x and y values between 0 and $2^N-1$ for $N=3$ (x values on columns and y values on rows).

  \begin{tabular}{ r | c c c c c c c c }
     & 0 & 1 & 2 & 3 & 4 & 5 & 6 & 7 \\ \hline
    0 & 0 & 1 & 1 & 1 & 1 & 1 & 1 & 1 \\
    1 & 0 & 0 & 1 & 1 & 1 & 1 & 1 & 1 \\
    2 & 0 & 0 & 0 & 1 & 1 & 1 & 1 & 1 \\
    3 & 0 & 0 & 0 & 0 & 1 & 1 & 1 & 1 \\
    4 & 0 & 0 & 0 & 0 & 0 & 1 & 1 & 1 \\
    5 & 0 & 0 & 0 & 0 & 0 & 0 & 1 & 1 \\
    6 & 0 & 0 & 0 & 0 & 0 & 0 & 0 & 1 \\
    7 & 0 & 0 & 0 & 0 & 0 & 0 & 0 & 0 \\
  \end{tabular}

  In general, for x and y values between 0 and $2^N-1$, GT(x,y) is 0 on the diagonal ($x=y$), 0 below the diagonal ($x<y$), and 1 above the diagnoal ($x>y$). Then, a fooling set of GT is the diagonal of the table,
  $F=\{(0,0), (1,1), (2,2), \ldots, (2^N-1,2^N-1)\}$. $F$ is a fooling set of GT because\\
  (1)For all $i, j$, $f(x_i,y_i)=f(x_j,y_j)=0$ and \\
  (2)For all $i>j$, $f(x_i, y_j) = 1 \neq f(x_i, y_i)=0$ 
  and for all $i<j$, $f(x_j, y_i) = 1 \neq f(x_i, y_i)=0$.
  Combining the two, for all $i \neq j$, $f(x_i, y_i) \neq f(x_i, y_j)$ or $f(x_i, y_i) \neq f(x_j, y_i)$. 

  The size of $F$ is $|F|=2^N$. The fooling set method states that for a function $f$ and a fooling set $F$ of size $l$, $D(f) \ge log(l)$. Applying this to GT, $D(GT) \ge log(|F|) = log(2^N) = N$. Therefore, the deterministic communication complexity of GT is at least $N$. 

\end{question}

\end{document}
