\documentclass{amsart}

\usepackage[vlined,linesnumbered]{algorithm2e}
\usepackage{enumitem}

\title{COSC-373: Homework 4}
\author{Lee Jiaen}
\author{Hyery Yoo}
\author{Alexander Lee}

\theoremstyle{definition}
\newtheorem{question}{Question}

\begin{document}

\maketitle

\begin{question}
  Each node $v$ executes as follows.

  Each round:
  \begin{enumerate}
    \item Pick a uniformly random color $c(v)$ from the range $1, 2, \ldots, 2
      \deg(v)$
    \item Send message $c(v)$ to all neighbors
    \item If $c(v) \neq c(u)$ for all neighbors $u$, output $c(v)$ and halt
  \end{enumerate}

  (Correctness) At each round a node $v$ picks a color $c(v)$ uniformly at
  random from the range $1, 2, \ldots, 2 \deg(v)$. Then, node $v$ sends its
  chosen color $c(v)$ to all its neighbors. Similarly, node $v$ would also
  receive colors from its active neighbors. If one of $v$'s neighbors previously
  halted while $v$ is still active, $v$ uses the last color it received from the
  neighbor. For each neighbor $u$ of node $v$, if $c(v) \neq c(u)$, the
  condition that neighboring vertices are assigned different colors is
  satisfied, so node $v$ outputs $c(v)$ and halts. Therefore, this protocol
  produces a proper $2 \Delta$ coloring of $G$.

  (Congestion) Since the protocol only requires nodes to send colors from the
  range $1, 2, \ldots, 2 \Delta$ to neighboring nodes, only messages of size
  $O(1)$ bits are being sent. Thus, the protocol is in the CONGEST model.

  (Runtime) To show that the protocol produces a proper $2 \Delta$ coloring of
  $G$ in $O(\log n)$, first define the random variable $X_v$ for whether a given
  node $v$ halts in a given round. Specifically, $X_v$ is 1 if node $v$ halts in
  the given round, and 0 otherwise. For a given round, node $v$'s neighbors
  could choose at most $\deg(v)$ distinct colors. We also know that node $v$
  chooses uniformly at random $2 \deg(v)$ from colors. As such, the probability
  that node $v$ chooses the same color as one of its neighbors, is at most
  $\frac{\deg(v)}{2 \deg(v)} = \frac{1}{2}$. It follows that the probability
  $\mathbb{P}(X_v)$ that node $v$ halts in a given round (i.e., choosing a
  different color from its neighbors) is
  \begin{align*}
    \mathbb{P}(X_v) &\ge 1 - \frac{1}{2} \\
    &= \frac{1}{2}.
  \end{align*}
  It follows that the expected value $\mathbb{E}(X_v)$ of node $v$ halting in a
  given round is
  \begin{align*}
    \mathbb{E}(X_v) &\ge 1 \cdot \frac{1}{2} + 0 \cdot \frac{1}{2} \\
    &= \frac{1}{2}.
  \end{align*}
  Next, define the random variable $X$ for the number of nodes that halt in a
  given round. Given the set of nodes $V$ for graph $G$, we have that
  \[
    X = \sum_{v \in V} X_v \enspace.
  \]
  Taking the expected value on both sides, we get
  \begin{align*}
    \mathbb{E}(X) &= \mathbb{E}\left(\sum_{v \in V} X_v\right) \\
    &= \sum_{v \in V} \mathbb{E}(X_v) \\
    &\ge \sum_{v \in V} \frac{1}{2} \\
    &= \frac{1}{2} n
  \end{align*}
  In expectation, at least $\frac{1}{2}$ of all nodes halt in each round. So if
  $n = n_0, n_1, \ldots, n_k$ are the number of active nodes after iterations
  $0, 1, \ldots, k$, we have that
  \[
    \mathbb{E}(n_k) \le \frac{1}{2^k} \cdot n.
  \]
  Taking $k = 4 \log n$, we get
  \begin{align*}
    \mathbb{E}(n_k) &\le \frac{1}{2^{4 \log n}} \cdot n \\
    &= \frac{1}{n^4} \cdot n \\
    &\le \frac{1}{n^2}.
  \end{align*}
  By Markov's Inequality, it follows that $\mathbb{P}(n_k \ge 1) \le
  \frac{1}{n^2}$. That is, after $4 \log n$ rounds, the probability that there
  are one or more active nodes is at most $\frac{1}{n^2}$. Therefore, the
  protocol produces a proper $2 \Delta$ coloring of $G$ in $O(\log n)$ rounds
  with high probability.
\end{question}

\newpage

\begin{question}
  TODO
\end{question}

\newpage

\begin{question}
  TODO
\end{question}

\end{document}
