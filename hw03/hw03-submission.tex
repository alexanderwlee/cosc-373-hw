\documentclass{amsart}

\usepackage[vlined,linesnumbered]{algorithm2e}
\usepackage{enumitem}

\title{COSC-373: Homework 2}
\author{Lee Jiaen}
\author{Hyery Yoo}
\author{Alexander Lee}

\theoremstyle{definition}
\newtheorem{question}{Question}

\begin{document}

\maketitle

\begin{question}
  TODO
\end{question}

\begin{question}
  TODO
\end{question}

\begin{question}
  \begin{enumerate}[label={(\alph*)}]
    \item We proceed with devising a greedy algorithm that is guaranteed to
      produce a proper $\Delta + 1$ coloring of $G$. The algorithm executes as
      follows:
      \begin{enumerate}[label={(\arabic*)}]
        \item Initialize an empty set $C$, which represents the set of colors.
        \item Assign any vertex in $V$ the color 1 and add 1 to $C$.
        \item Choose an uncolored vertex $v \in V$. Assign $v$ the smallest
          color in $C$ that has not been assigned to $v$'s colored neighbors. If
          $v$'s colored neighbors have been assigned all the colors in $C$, add
          a new color $\max\{C\} + 1$ to $C$ and assign it to $v$.
        \item Repeat step (2) until all vertices are colored.
      \end{enumerate}

      Since the algorithm assigns an uncolored vertex $v \in V$ with the lowest
      color in $C$ that has not been assigned to $v$'s neighbors or assigns $v$
      with a new color not yet in $C$, no two neighboring vertices are assigned
      the same color. Furthermore, because the algorithm only adds a new color
      to $C$ when all neighbors of $v$ have been assigned all the colors in $C$
      and $v$ has at most $\Delta$ neighbors, the algorithm adds at most $\Delta
      + 1$ colors to $C$. Thus, only $\Delta + 1$ colors are required.
      Therefore, the algorithm produces a proper $\Delta + 1$ coloring of $G$.

    \item Consider the graph $G_k = (V_k, E_k)$, where $V_k = \{1, 2, \ldots,
      \Delta + 1\}$ and $E_k = \{(1, 2), (1, 3), \ldots, (1, \Delta + 1)\} \cup
      \{(2, 3), (3, 4), \ldots, (\Delta, \Delta + 1)\}$. Here, vertex 1 has a
      degree equal to $\Delta$ and every other vertex in $G_k$ has a degree of
      at most $\Delta$. Clearly, the maximum degree is $\Delta$. Using colors
      from the range $1, 2, \ldots, \Delta$, regardless of how colors are
      assigned to the vertices in $G_k$, at least two neighboring vertices will
      be assigned the same colors. Thus, $G_k$ does not admit a proper $\Delta$
      coloring.

    \item Each phase $i$:
      \begin{itemize}
        \item Execute protocol $\Pi$
        \item If node is in the MIS found from $\Pi$
          \begin{itemize}
            \item Output $i$ and halt
          \end{itemize}
      \end{itemize}

      (Correctness) In each phase $i$, protocol $\Pi$ finds a MIS in the network
      of remaining nodes that have not halted. The nodes found in the MIS from
      phase $i$ output color $i$. Neighboring nodes will not output the same
      color since they belong to different maximal independent sets.

      (Runtime) Since each node requires at most $\Delta + 1$ phases to output a
      color $i$ and each phase runs in $T(n)$ rounds, the protocol thus computes
      a property $\Delta + 1$ coloring of $G$ in $O((\Delta + 1) T(n)) =
      O(\Delta T(n) + T(n)) = O(\Delta T(n))$ rounds.
  \end{enumerate}
\end{question}

\end{document}
