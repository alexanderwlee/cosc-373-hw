\documentclass{amsart}

\usepackage[vlined,linesnumbered]{algorithm2e}
\usepackage{enumitem}

\title{COSC-373: Homework 4}
\author{Lee Jiaen}
\author{Hyery Yoo}
\author{Alexander Lee}

\theoremstyle{definition}
\newtheorem{question}{Question}

\begin{document}

\maketitle

\begin{question}
  Each node $v$ executes as follows.

  Each round:
  \begin{enumerate}
    \item Pick a uniformly random color $c(v)$ from the range $1, 2, \ldots, 2
      \deg(v)$
    \item Send message $c(v)$ to all neighbors
    \item If $c(v) \neq c(u)$ for all neighbors $u$, output $c(v)$ and halt
  \end{enumerate}

  (Correctness) At each round a node $v$ picks a color $c(v)$ uniformly at
  random from the range $1, 2, \ldots, 2 \deg(v)$. Then, node $v$ sends its
  chosen color $c(v)$ to all its neighbors. Similarly, node $v$ would also
  receive colors from its active neighbors. If one of $v$'s neighbors has
  already halted, $v$ uses the last color it received from the neighbor. For
  each neighbor $u$ of node $v$, if $c(v) \neq c(u)$, the condition that
  neighboring vertices are assigned different colors is satisfied, so node $v$
  outputs $c(v)$ and halts. Therefore, this protocol produces a proper $2
  \Delta$ coloring of $G$.

  (Congestion) Since the protocol only requires nodes to send colors from the
  range $1, 2, \ldots, 2 \Delta$ to neighboring nodes, only messages of size
  $O(1)$ bits are being sent. Thus, the protocol is in the CONGEST model.

  (Runtime) To show that the protocol produces a proper $2 \Delta$ coloring of
  $G$ in $O(\log n)$ rounds, first define the random variable $X_v$ for whether
  a node $v$ halts in a given round. Specifically, $X_v$ is 1 if node $v$ halts
  in the given round, and 0 otherwise. For a given round, since node $v$ has
  $\deg(v)$ neighbors, its neighbors can choose at most $\deg(v)$ distinct
  colors from each other. We also know that node $v$ chooses uniformly at random
  from $2 \deg(v)$ colors. As such, the probability that node $v$ chooses the
  same color as one of its neighbors, is at most $\frac{\deg(v)}{2 \deg(v)} =
  \frac{1}{2}$. It follows that the probability $\mathbb{P}(X_v)$ that node $v$
  halts in a given round (i.e., choosing a different color from its neighbors)
  is
  \begin{align*}
    \mathbb{P}(X_v) &\ge 1 - \frac{1}{2} \\
    &= \frac{1}{2}.
  \end{align*}
  Hence, the expected value $\mathbb{E}(X_v)$ of node $v$ halting in a given
  round is
  \begin{align*}
    \mathbb{E}(X_v) &\ge 1 \cdot \frac{1}{2} + 0 \cdot \frac{1}{2} \\
    &= \frac{1}{2}.
  \end{align*}
  Next, define the random variable $X$ for the number of nodes that halt in a
  given round. Given the set of nodes $V$ for graph $G$, we have that
  \[
    X = \sum_{v \in V} X_v \enspace.
  \]
  Taking the expected value on both sides, we get
  \begin{align*}
    \mathbb{E}(X) &= \mathbb{E}\left(\sum_{v \in V} X_v\right) \\
    &= \sum_{v \in V} \mathbb{E}(X_v) \\
    &\ge \sum_{v \in V} \frac{1}{2} \\
    &= \frac{1}{2} n
  \end{align*}
  Thus, in expectation, at least $\frac{1}{2}$ of all nodes halt in each round.
  If $n = n_0, n_1, \ldots, n_k$ are the number of active nodes after rounds
  $0, 1, \ldots, k$, we have that
  \[
    \mathbb{E}(n_k) \le \frac{1}{2^k} \cdot n.
  \]
  Taking $k = 4 \log n$, we get
  \begin{align*}
    \mathbb{E}(n_k) &\le \frac{1}{2^{4 \log n}} \cdot n \\
    &= \frac{1}{n^4} \cdot n \\
    &= \frac{1}{n^3}.
  \end{align*}
  By Markov's Inequality, it follows that $\mathbb{P}(n_k \ge 1) \le
  \frac{1}{n^3}$. That is, after $4 \log n$ rounds, the probability that there
  are one or more active nodes is at most $\frac{1}{n^3}$. Therefore, the
  protocol produces a proper $2 \Delta$ coloring of $G$ in $O(\log n)$ rounds
  with high probability.
\end{question}

\newpage

\begin{question}
\begin{enumerate}
For F = \{($A_1$, $A^c_1$) $|$ A $\subseteq$ \{1,2,... N\}\} to be a fooling set for function DISJ, we must prove the following: 
    \item For all i,j we have DISJ($A_i$, $A^c_i$) = DISJ($A_j$, $A^c_j$)
    \item For all i,j  we have DISJ($A_i$, $A^c_i$) $\neq$ DISJ($A_i$, $A^c_j$) OR  DISJ($A_i$, $A^c_i$) $\neq$ DISJ($A_j$, $A^c_i$)
\\To prove part 1, suppose we have an arbitrary complementary pair of sets in F, ($A_i$, $A^c_i$). Since F is the set of all pairs of sets ($A$, $A^c$) where $A^c$ is the complement of $A$, $A^c$ is the set of elements from {1,2,...,N} that are NOT contained in $A$. Therefore, $A$ and $A^c$ have no elements in common and for all i, any pair of sets in F would return DISJ($A_i$, $A^c_i$) = 1. Thus, for all i and j, we have DISJ($A_i$, $A^c_i$) = DIDJ($A_j$, $A^c_j$) = 1.  
\\To prove part 2,  suppose we have an arbitrary complementary pair of sets in F, ($A_i$, $A^c_i$), and another arbitrary complementary pair of sets in F, ($A_j$, $A^c_j$). Thus, we have three situations that could occur:
    \item $A_j$ shares elements with only $A_i$ but does not equal $A_i$ 
    \begin{enumerate}[label={(\alph*)}]
        \item DISJ($A_i$, $A^c_i$) = 1, DISJ($A_i$, $A^c_j$) = 0, DISJ($A_j$, $A^c_i$) = 1. Thus, DISJ($A_i$, $A^c_i$) $\neq$ DISJ($A_i$, $A^c_j$)
    \end{enumerate}
    \item $A_j$ shares elements with only $A^c_i$ but does not equal $A^c_i$
    \begin{enumerate}[label={(\alph*)}]
    \item DISJ($A_i$, $A^c_i$) = 1, DISJ($A_i$, $A^c_j$) = 0, DISJ($A_j$, $A^c_i$) = 0. Thus,  DISJ($A_i$, $A^c_i$) $\neq$ DISJ($A_i$, $A^c_j$) and  DISJ($A_i$, $A^c_i$) $\neq$ DISJ($A_j$, $A^c_i$)
    \end{enumerate}
    \item Aj shares elements with both Ai and Aci
    \begin{enumerate}[label={(\alph*)}]
    \item DISJ($A_i$, $A^c_i$) = 1, DISJ($A_i$, $A^c_j$) = 0, DISJ($A_j$, $A^c_i$) = 0. Thus,  DISJ($A_i$, $A^c_i$) $\neq$ DISJ($A_i$, $A^c_j$) and  DISJ($A_i$, $A^c_i$) $\neq$ DISJ($A_j$, $A^c_i$)
    \end{enumerate}
Thus, for all i,j  we have DISJ($A_i$, $A^c_i$) $\neq$ DISJ($A_i$, $A^c_j$) OR  DISJ($A_i$, $A^c_i$) $\neq$ DISJ($A_j$, $A^c_i$) and we prove that F is a fooling set of the function DISJ. 
\end{enumerate}
\end{question}

\newpage

\begin{question}
  TODO
\end{question}

\end{document}
